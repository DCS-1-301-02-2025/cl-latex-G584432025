\documentclass[a4paper,11pt,dvipdfmx]{ujarticle}
% パッケージ
\usepackage{graphicx}
\usepackage{url}
% レイアウト指定を記述したファイルの読み込み
\input{layout}

% タイトルと氏名を変更せよ.
\title{日本におけるデジタル化の状況}
\author{G584432025 佐浦 康介}

\begin{document}

\maketitle %ここにタイトルが入る

\section{デジタル競争力ランキング   }
国際経営開発研究所(IMD)の調査\cite{IMD}によると、日本のデジタル競争力のランキングは図\ref{fig:順位}に示すよ
うに、調査対象の64カ国中、総合で28位、技術分野で30位となっている。

\begin{figure}[htbp]
    \centering
    \includegraphics[width=0.7\linewidth]{fig41.png}
    \caption{デジタル競争力ランキング(64カ国中)}
    \label{fig:順位}
\end{figure}

\newpage
\section{ブロードバンドの整備状況}
OECDによるブロードバンド回線の普及に関する調査\cite{OECD}.
によると、表\ref{tbl:加入者数}に示すように、日本における 100人あたりのモバイルブロードバンドの加入者数は190.5で、第1位になっている。2位はエストニア
で、3位米国と続く。

\begin{table}[htbp]
    \centering
    \caption{モバイルブロードバンドの加入者数(100人あたり)}
    \label{tbl:加入者数}

    \begin{tabular}{|l|r|l|}\hline
        順位 & 国名 & 加入者数 \\
        \hline
        1位 & 日本 & 190.5 \\
        \hline
        2位 & エストニア & 179.9 \\
        \hline
        3位 & 米国 & 169.0 \\
        \hline
        4位 & フィンランド & 157.0 \\
        \hline
        5位 & デンマーク & 141.7 \\
        \hline
        6位 & ラトビア & 141.6 \\
        \hline 
        7位 & イスラエル & 139.9 \\
        \hline
        8位 & オランダ & 133.7 \\
        \hline
        9位 & ポーランド & 131.3 \\
        \hline
        10位 & スウェーデン & 127.2 \\
        \hline
    \end{tabular}
\end{table}

\section{考察}
\begin{itemize}
  \item 日本はモバイル通信環境がすごく整ってるのに、それを活かしたサービスやシステムの開発が他国より遅れているように感じた。
  \item インフラだけじゃなくて、技術を実際に使える人や、それを育てる教育の仕組みも強化していく必要があると思った。
  \item 日本はスマホの回線は最強なのに、それを活かすアプリとかサービスは海外のもの(Google、Instagram、ChatGPTとか)ばっかり使ってるのがちょっと悔しい。せっかくの回線、もっと「日本発」で世界にウケるようなサービスを生み出したいって思った。
\end{itemize}

  
% 本文(1)
%  参考文献の参照: \cite{}
%  図番号の参照: \ref{}
% を使う
% 文献データベースのキーワードは oecd と imd
% になっている.

% 図の挿入
% \includegraphics{}
% を
% \begin{figure}[htbp]
% \end{figure}
% で囲み
% \caption{}
% で図のタイトルを入れる.
% \label{}
% を使って図番号が参照できるようにする
% また,
% \centering
% で図が中央に来るようにする

% ーーー
% 節見出し(2)

% 本文(2)

% 表の挿入
% \begin{tabular}
% \end{tabular}    
% による表の記述を 
% \begin{table}[htbp]
% \end{table}
% で囲み
% \caption{}
% で表のタイトルを入れる.
% \label{}
% を使って表番号が参照できるようにする
% また,
% \centering
% で表が中央に来るようにする

% ーーー
% 見出し(3)
% 考察
%
% \begin{itemize}
% \end{itemize}
% を使って箇条書きで記述する

% ここに参考文献が入る
%
\bibliographystyle{junsrt}
\bibliography{exercise.bib}

\end{document}